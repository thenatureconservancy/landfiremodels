% Options for packages loaded elsewhere
\PassOptionsToPackage{unicode}{hyperref}
\PassOptionsToPackage{hyphens}{url}
%
\documentclass[
]{article}
\usepackage{lmodern}
\usepackage{amssymb,amsmath}
\usepackage{ifxetex,ifluatex}
\ifnum 0\ifxetex 1\fi\ifluatex 1\fi=0 % if pdftex
  \usepackage[T1]{fontenc}
  \usepackage[utf8]{inputenc}
  \usepackage{textcomp} % provide euro and other symbols
\else % if luatex or xetex
  \usepackage{unicode-math}
  \defaultfontfeatures{Scale=MatchLowercase}
  \defaultfontfeatures[\rmfamily]{Ligatures=TeX,Scale=1}
\fi
% Use upquote if available, for straight quotes in verbatim environments
\IfFileExists{upquote.sty}{\usepackage{upquote}}{}
\IfFileExists{microtype.sty}{% use microtype if available
  \usepackage[]{microtype}
  \UseMicrotypeSet[protrusion]{basicmath} % disable protrusion for tt fonts
}{}
\makeatletter
\@ifundefined{KOMAClassName}{% if non-KOMA class
  \IfFileExists{parskip.sty}{%
    \usepackage{parskip}
  }{% else
    \setlength{\parindent}{0pt}
    \setlength{\parskip}{6pt plus 2pt minus 1pt}}
}{% if KOMA class
  \KOMAoptions{parskip=half}}
\makeatother
\usepackage{xcolor}
\IfFileExists{xurl.sty}{\usepackage{xurl}}{} % add URL line breaks if available
\IfFileExists{bookmark.sty}{\usepackage{bookmark}}{\usepackage{hyperref}}
\hypersetup{
  pdftitle={Welcome page},
  pdfauthor={Megan Dettenmaier},
  hidelinks,
  pdfcreator={LaTeX via pandoc}}
\urlstyle{same} % disable monospaced font for URLs
\usepackage[margin=1in]{geometry}
\usepackage{graphicx,grffile}
\makeatletter
\def\maxwidth{\ifdim\Gin@nat@width>\linewidth\linewidth\else\Gin@nat@width\fi}
\def\maxheight{\ifdim\Gin@nat@height>\textheight\textheight\else\Gin@nat@height\fi}
\makeatother
% Scale images if necessary, so that they will not overflow the page
% margins by default, and it is still possible to overwrite the defaults
% using explicit options in \includegraphics[width, height, ...]{}
\setkeys{Gin}{width=\maxwidth,height=\maxheight,keepaspectratio}
% Set default figure placement to htbp
\makeatletter
\def\fps@figure{htbp}
\makeatother
\setlength{\emergencystretch}{3em} % prevent overfull lines
\providecommand{\tightlist}{%
  \setlength{\itemsep}{0pt}\setlength{\parskip}{0pt}}
\setcounter{secnumdepth}{-\maxdimen} % remove section numbering

\title{Welcome page}
\author{Megan Dettenmaier}
\date{7/6/2020}

\begin{document}
\maketitle

\begin{your-class}

your content

your content

your content

\end{your-class}

\hypertarget{hello-and-welcome}{%
\subsection{Hello and Welcome}\label{hello-and-welcome}}

Hello and welcome to the LANDFIRE vegetation modeling site. We created
this site to help you harness the potential of the modeling framework to
accomplish your management and restoration objectives. We recognize the
importance of having the best resources to accomplish your
goals\ldots and here at LANDFIRE we are continually improving our
products to get you the data you need.

\hypertarget{so-what-will-you-learn-heres-just-a-sample}{%
\paragraph{So, what will you learn? Here's just a
sample:}\label{so-what-will-you-learn-heres-just-a-sample}}

\begin{enumerate}
\def\labelenumi{\arabic{enumi}.}
\tightlist
\item
  We'll discuss how LANDFIRE historical vegetation models can jumpstart
  your research, management planning and understanding of both current
  and future ecosystems
\item
  We'll help you navigate the Syncrosim biophysical vegetation modeling
  framework
\item
  And finally, we'll offer real life, on the ground examples of the
  models in action and demonstrate how modeling perturbations in the
  environment can inform and direct effective management actions
\end{enumerate}

\hypertarget{why-integrate-this-into-your-current-work}{%
\paragraph{Why integrate this into your current
work?}\label{why-integrate-this-into-your-current-work}}

This framework provides a national baseline for assessing current
vegetation conditions while accounting for historical ecology.
Integrating this knowledge with our current projections of climate and
other factors can has the potential to improve the way we manage land
while providing some ecological context. This modeling framework can
help you making predictions about future landscape conditions. For
example, you may be interested in understanding how an insect invasion
will impact a specific region, with the right data and a little bit of
time, this program can help you make predictions and plan for the
future.

\hypertarget{what-you-will-learn}{%
\paragraph{What you will learn}\label{what-you-will-learn}}

\begin{enumerate}
\def\labelenumi{\arabic{enumi}.}
\tightlist
\item
  We'll show you how to accessing the data on landfire.gov
\item
  We'll walk you through how to run and adapt the models
\item
  And finally we'll show you how to interpret the results
\end{enumerate}

\hypertarget{lets-get-started}{%
\paragraph{Let's get started}\label{lets-get-started}}

\begin{enumerate}
\def\labelenumi{\arabic{enumi}.}
\item
  To get the process rolling you need to install syncrosim on your
  machine - but before you do that, you need to register with syncrosim,
  go to - \url{https://syncrosim.com/}
\item
  Then you'll go to download (that's emailed to you (RIGHT?)), and find
  the version of Syncrosim that's appropriate for your hardware.
\item
  Next you'll head on over to the LANDFIRE site and download the Master
  ST-Sim database.
  \url{https://www.landfire.gov/national_veg_models_op2.php}
\item
  And finally, open Syncrosim, go to file, open library, locate the
  unzipped Master St-Sim database that you just downloaded. Once you do
  that, you're ready to get started.
\end{enumerate}

\hypertarget{use-this-website}{%
\paragraph{Use this website}\label{use-this-website}}

We designed this website as a resource for both new and seasoned BpS
users. You can skip ahead to the useful videos and sections, or navitate
through each tab and section as it suites you. We stuck with one example
in these tutorials to remain consistent and help you follow our model
``story''. We'll be exploring the Sage Steppe BpS throughout this site.

Either way, we hope you find our website useful and most importantly, we
hope you use these BpS models to their fullest extent - Your next step
is to find and organize your model. (insert link)

\end{document}
